\documentclass[10pt]{article}

\usepackage{fullpage}
\usepackage{setspace}
\usepackage{parskip}
\usepackage{titlesec}
\usepackage[section]{placeins}
\usepackage{xcolor}
\usepackage{breakcites}
\usepackage{lineno}
\usepackage{hyphenat}





\PassOptionsToPackage{hyphens}{url}
\usepackage[colorlinks = true,
            linkcolor = blue,
            urlcolor  = blue,
            citecolor = blue,
            anchorcolor = blue]{hyperref}
\usepackage{etoolbox}
\makeatletter
\patchcmd\@combinedblfloats{\box\@outputbox}{\unvbox\@outputbox}{}{%
  \errmessage{\noexpand\@combinedblfloats could not be patched}%
}%
\makeatother


\usepackage{natbib}




\renewenvironment{abstract}
  {{\bfseries\noindent{\abstractname}\par\nobreak}\footnotesize}
  {\bigskip}

\titlespacing{\section}{0pt}{*3}{*1}
\titlespacing{\subsection}{0pt}{*2}{*0.5}
\titlespacing{\subsubsection}{0pt}{*1.5}{0pt}


\usepackage{authblk}


\usepackage{graphicx}
\usepackage[space]{grffile}
\usepackage{latexsym}
\usepackage{textcomp}
\usepackage{longtable}
\usepackage{tabulary}
\usepackage{booktabs,array,multirow}
\usepackage{amsfonts,amsmath,amssymb}
\providecommand\citet{\cite}
\providecommand\citep{\cite}
\providecommand\citealt{\cite}
% You can conditionalize code for latexml or normal latex using this.
\newif\iflatexml\latexmlfalse
\providecommand{\tightlist}{\setlength{\itemsep}{0pt}\setlength{\parskip}{0pt}}%

\AtBeginDocument{\DeclareGraphicsExtensions{.pdf,.PDF,.eps,.EPS,.png,.PNG,.tif,.TIF,.jpg,.JPG,.jpeg,.JPEG}}

\usepackage[utf8]{inputenc}
\usepackage[english]{babel}



\usepackage{float}








\begin{document}

\title{Project Proposal - ETF price forecaster}



\author[1]{Marco Sampaio}%
\affil[1]{Affiliation not available}%


\vspace{-1em}




  \date{January 11, 2025}


\begingroup
\let\center\flushleft
\let\endcenter\endflushleft
\maketitle
\endgroup





\selectlanguage{english}
\begin{abstract}
We propose a project to develop a machine learning model to forecast
Exchange Trade Fund (ETF) indices. The goal of the project is not only
to develop a forecaster with a good performance but also to deploy an
interactive dashboard on AWS infrastructure with daily model
retraining.~%
\end{abstract}%



\sloppy


\hypertarget{domain-background}{%
\section*{Domain Background}}

{\label{113420}}

Exchange-Traded Funds (ETFs)~\hyperref[csl:1]{[1]}; \hyperref[csl:2]{[2]} are investment
instruments consisting of a basket of securities--- such as stocks,
bonds, or commodities---designed to track the performance of a specific
index, sector, or asset class. ETFs are traded on stock exchanges,
allowing investors to buy and sell shares throughout the day at market
prices.~

ETFs are an excellent tool for individuals to manage long-term savings
due to their ability to provide diversified exposure to various asset
classes at a low cost \hyperref[csl:3]{[3]}. By investing in ETFs that
track broad market indices, such as the S\&P 500, individuals can
achieve consistent growth over time, benefiting from market trends and
compounding returns at low management fees and tax efficiency, making
them a popular choice for retirement planning and other financial goals.

\hypertarget{problem-statement}{%
\section*{Problem Statement}}

{\label{942564}}

In this project, we propose to develop an interactive dashboard that
shows the evolution of a selection of long term high performing ETF
indices, together with Machine Learning (ML) model based forecasts (in a
relatively short time window e.g., 1 month). We have in mind the use
case of an individual who:

\begin{itemize}
\tightlist
\item
  has some idea on the domains they want to invest in (e.g., they may be
  thinking about keeping half their savings in an index that tracks the
  overall market, like S\&P 500, and distribute the remainder in the
  tech sector, robotics, health, etc.);
\item
  is interested in regularly investing their savings in such funds
  (e.g., several times a year);
\item
  wants to decide the best time during a period (e.g., one month) to
  buy, if they are investing, or sell, if they are withdrawing their
  funds to use them.
\end{itemize}

Thus, overall, such an individual is not interested in having a real
time forecast of what's happening to the market throughout the day, but
it is interested in the overall trend in a window of a couple of week to
make an informed decision on how low they should set their buying price,
or how high their selling price so that their order gets executed within
the time frame they have set while not executing it at a very
unfavorable market fluctuation.

More specifically, the problem can be stated as follows. Given a
historical time series dataset of with a set of closing
prices~\(\left\{x_{i,t}\right\}\) for the ETF index~\(i\) on
business day~\(t\) we want to predict the future
prices~\(\left\{\hat{x}_{i, t}\right\}\) for all the indices.~

\hypertarget{datasets-inputs-and-solution-statement}{%
\section*{Datasets,~ Inputs and Solution
Statement}}

{\label{273725}}

Currently there is a wide array of free data resources to obtain time
series data for finance. Yahoo finance, in particular, is a widely used
source.~ We aim to leverage the following two data sources:

\begin{itemize}
\tightlist
\item
  A publicly available Kaggle dataset~\hyperref[csl:4]{[4]} that has been
  scrapped from yahoo finance from thousands of ETF symbols in the US.
  This will be useful for exploratory data analysis and allow to start
  iterating ideas fast.
\item
  the yfinance python libary~\hyperref[csl:5]{[5]} allows to download up to
  date Yahoo finance data. This will allow us to get price data on a
  daily basis in order to be able to retrain models periodically.~
\end{itemize}

The quantity we will be interested in forecasting is the closing price
on each day, which is available in either of these data sources.

To solve this problem and predict the target outputs, the input of the
model can be in many forms depending on the model used, namely:

\begin{itemize}
\tightlist
\item
  Previous values of the time series before the prediction
  time~\(t\);~
\item
  Manually engineered features~ based on the past series values (e.g.,
  via rolling window statistics);
\item
  Features extracted from the date (e.g., week day, month, etc.);
\item
  etc.
\end{itemize}

Which such types of inputs are used depends largely on the type of model
used.~

\hypertarget{benchmark-model}{%
\section*{Benchmark Model}}

{\label{145318}}

The simplest benchmark models that can be defined in the context of time
series prediction are:

\begin{itemize}
\tightlist
\item
  \textbf{Predict no change in price:}~Since market prices fluctuate up
  and down the overall mean tends to drift slowly, so we expect this to
  be good reference baseline (if we do worse than such a simple model
  then we have a clear red flag).
\item
  \textbf{Predict based on a long term average return rate:~}In our use
  case, the investor is mostly interested in the overall trend to ensure
  they invest at a favorable time. The return rate computed, e.g., on a
  few weeks or a few months is typically a good indication of the trend
  of the evolution of the prices so it is expected to provide a decent
  baseline.~
\end{itemize}

\hypertarget{evaluation-metrics}{%
\section*{Evaluation Metrics}}

{\label{323921}}

We propose two types of metrics to evaluate the models:

\begin{itemize}
\tightlist
\item
  \textbf{Loss for training:} In regression problems, the typical metric
  is root mean squared error~\hyperref[csl:6]{[6]}, which is the average of
  the squared residuals of the prediction at each step of the time
  series. Out of the box models in standard libraries for regression
  typically use this as the default loss for fitting.
\item
  \textbf{Model evaluation} (validation and test set): To evaluate the
  quality of the forecasts we propose to focus on Median Absolute
  Percentage Error (MedianAPE) for each series - defined as the Median
  of the absolute residuals. This will be used in validation to select~
  the best model hyperparameters.~ We choose the median because it is
  less sensitive to outlier differences, which are common to happen in
  finance due to isolated unusual market events, that deviate a lot from
  the mean behavior and are very unpredictable (so we do not aim to be
  able to predict them).~
\end{itemize}

\hypertarget{project-design}{%
\section*{Project Design}}

{\label{381510}}

To execute the project we will adopt the stages described in the next
sections

\hypertarget{exploratory-data-analysis-data-preparation-and-cleaning}{%
\subsection*{Exploratory data analysis, data preparation and
cleaning}}

{\label{771372}}

In this stage we will answer the following questions:

\begin{itemize}
\tightlist
\item
  Which ETFs do we want to focus on? Is the data clean and complete?
\item
  How do the time series look like? Should we model the price directly?
  Or the daily returns?
\item
  Are there any trends, seasonality, time correlation etc. to take into
  account and that we can exploit in modeling?
\end{itemize}

\hypertarget{model-development-evaluation}{%
\subsection*{Model development \&
evaluation}}

{\label{177371}}

The next step will be to develop and evaluate models based on historical
data. We will split the data temporally in train, validation and test.
The test set will be held out for final evaluation of only a few short
listed models whereas the validation set will be used to select the
hyper-parameters of each type of model. In principle we can explore
various types of modelling approaches:

\begin{itemize}
\tightlist
\item
  ARIMA models~\hyperref[csl:7]{[7]}: These are classic time series models
  that include autoregressive terms (linearly regressing on previous
  series values), moving average terms (linearly regressing on residual
  errors) and time series differencing.
\item
  Feature based non-linear ML models: This simply consists of taking an
  out of the box model such as a gradient boosting decision trees
  model~\hyperref[csl:8]{[8]} and feed it various features that describe
  the history of the time series (statistics, series values etc.) to
  learn non-linear dependencies.
\item
  Non-linear autoregressive~ neural network models~\hyperref[csl:9]{[9]}:
  RNN models such as LSTMs provide another approach to modeling time
  series non-linearities which does not require explicit feature
  engineering. These are typically more difficult to train though.
\end{itemize}

\hypertarget{dashboard-app-development-deployment}{%
\subsection*{Dashboard app development \&
deployment}}

{\label{177371}}

The last step of the project will consist of developing a code framework
to manage the deployment of the interactive dashboard app. We aim to
automate all steps of the process so that:

\begin{itemize}
\tightlist
\item
  The model is retrained daily to produce up to date forecasts without
  any manual intervention
\item
  The dashboard app is deployed in an auto scaling group and available
  at a public URL and allows for a user to select the ETFs they want to
  visualize. The Dashboard will allow for the user to clearly see the
  forecasts and the historical data and it will allow them to zoom
  specific data periods and consult the specific values of prices at
  specific dates.
\end{itemize}

Below we present two diagrams of what we envision for these two
components and how they should interact:\selectlanguage{english}
\begin{figure}[H]
\begin{center}
\includegraphics[width=1.00\columnwidth]{figures/Solution-Architecture/Solution-Architecture}
\caption{{Early concept for the deployment Architecture
{\label{582012}}%
}}
\end{center}
\end{figure}\selectlanguage{english}
\begin{figure}[H]
\begin{center}
\includegraphics[width=0.49\columnwidth]{figures/Dashboard-mockup/Dashboard-mockup}
\caption{{Early concept for the structure of the dashboard
{\label{140820}}%
}}
\end{center}
\end{figure}

\selectlanguage{english}
\FloatBarrier
\section*{References}\sloppy
\phantomsection
\label{csl:1}[1]``{Exchange-Traded Fund (ETF): How to Invest and What It Is}''. \url{https://www.investopedia.com/terms/e/etf.asp.}

\phantomsection
\label{csl:2}[2]``{Exchange-traded fund - Wikipedia}''. \url{https://en.wikipedia.org/wiki/Exchange-traded_fund.}

\phantomsection
\label{csl:3}[3]``{What are exchange traded funds (ETFs)? | Vanguard}''. \url{https://investor.vanguard.com/investor-resources-education/etfs/what-is-an-etf.}

\phantomsection
\label{csl:4}[4]``{US Funds dataset from Yahoo Finance}''. \url{https://www.kaggle.com/datasets/stefanoleone992/mutual-funds-and-etfs.}

\phantomsection
\label{csl:5}[5]``{yfinance}''. \url{https://pypi.org/project/yfinance/.}

\phantomsection
\label{csl:6}[6]``{root_mean_squared_error}''. \url{https://scikit-learn.org/stable/modules/generated/sklearn.metrics.root_mean_squared_error.html.}

\phantomsection
\label{csl:7}[7]``{Autoregressive integrated moving average - Wikipedia}''. \url{https://en.wikipedia.org/wiki/Autoregressive_integrated_moving_average.}

\phantomsection
\label{csl:8}[8]``{lightgbm.LGBMRegressor — LightGBM 4.5.0.99 documentation}''. \url{https://lightgbm.readthedocs.io/en/latest/pythonapi/lightgbm.LGBMRegressor.html.}

\phantomsection
\label{csl:9}[9]``{Recurrent neural network - Wikipedia}''. \url{https://en.wikipedia.org/wiki/Recurrent_neural_network.}

\phantomsection
\label{csl:0}[10]``{Exchange-traded fund – Wikipédia, a enciclopédia livre}''. \url{https://pt.wikipedia.org/wiki/Exchange-traded_fund.}

\end{document}
